\documentclass[titlepage,a4paper]{article}

\usepackage{a4wide}
\usepackage[colorlinks=true,linkcolor=black,urlcolor=blue,bookmarksopen=true]{hyperref}
\usepackage{bookmark}
\usepackage{fancyhdr}
\usepackage[spanish]{babel}
\usepackage[utf8]{inputenc}
\usepackage[T1]{fontenc}
\usepackage{graphicx}
\usepackage{float}
\usepackage{hyperref}

\graphicspath{ {imagenes/} }  % Las imagenes las reubique en la carpeta imagenes

\pagestyle{fancy} % Encabezado y pie de página
\fancyhf{}
\fancyhead[L]{TP1}
\fancyhead[R]{Algoritmos y Programación III - FIUBA}
\renewcommand{\headrulewidth}{0.4pt}
\fancyfoot[C]{\thepage}
\renewcommand{\footrulewidth}{0.4pt}

\begin{document}
\begin{titlepage} % Carátula
	\hfill\includegraphics[width=6cm]{logofiuba.jpg}
    \centering
    \vfill
    \Huge \textbf{Trabajo Práctico 1 — Smalltalk}
    \vskip2cm
    \Large [7507/9502] Algoritmos y Programación III\\
    Curso 2 \\ % Curso 1 para el de la tarde y 2 para el de la noche
    Primer cuatrimestre de 2020 
    \vfill
    \begin{tabular}{ | l | l | } % Datos del alumno
      \hline
      Alumno: & Paredes Ramirez, Luis José\\ \hline
      Número de padrón: & 104851 \\ \hline
      Email: & lparedesr@fi.uba.ar \\ \hline
  	\end{tabular}
    \vfill
    \vfill
\end{titlepage}

\tableofcontents % Índice general
\newpage

\section{Introducción}\label{sec:intro}
  El presente informe reune la documentacion del primer trabajo practico de la materia
  Algoritmos y Programación III que consiste en desarrollar un sistema que permita encontrar el mejor 
  presupuesto de un pintor de acuerdo a las necesidades de los clientes. \newline

  Este trabajo es desarrollado usando los conceptos del paradigma de la programación orientación a objetos
  utilizando el sistema Smalltalk.


% Deberá contener explicaciones de cada uno de los supuestos que el alumno haya tenido que adoptar a partir de 
% situaciones que no estén contempladas en la especificación.
\section{Supuestos}\label{sec:supuestos}

Los siguientes supuestos se tomarán como constantes que siempre se cumpliran en el trabajo practico.

  \subsection{Pincel}
    \begin{itemize}
      \item El pintor que utiliza Pincel siempre tarda 2 horas en pintar un $M^2$.
      \item El pintor que utiliza Pincel siempre gastara 4 litros por $M^2$.
      \item El pintor que utiliza Pincel aplica un descuento del 50\% en la mano de obra para proyectos mayores a 40 $M^2$.
    \end{itemize}

  \subsection{Rodillo}
    \begin{itemize}
      \item El pintor que utiliza Rodillo siempre tarda 1 hora en pintar un $M^2$.
      \item El pintor que utiliza Rodillo siempre gastara 5 litros por $M^2$.
      \item El pintor que utiliza Rodillo no aplica descuento.
    \end{itemize}

  \subsection{Pintura}
    \begin{itemize}
      \item Pintura Alba siempre requiere 1 mano con pincel o 1 mano con rodillo.
      \item Pintura Venier siempre requiere 2 manos con pincel o 1 mano con rodillo.
    \end{itemize}

  \subsection{Pintor}
    \begin{itemize}
      \item Se asume que siempre se cargaran pintores que no tengan nombres repetidos.
    \end{itemize}
  

% Uno o varios diagramas de clases mostrando las relaciones estáticas entre las clases. 
% Puede agregarse todo el texto necesario para aclarar y explicar su diseño. 
%Recuerden que la idea de todo el documento es que quede documentado y entendible cómo está implementada la solución.
\newpage
\section{Diagramas de clase}\label{sec:diagramasdeclase}
%Uno o varios diagramas de clases mostrando las relaciones estáticas entre las clases.  
%Puede agregarse todo el texto necesario para aclarar y explicar su diseño. 
%Recuerden que la idea de todo el documento es que quede documentado y entendible cómo está implementada la solución. 
%Todos los diagramas tienen que estar embebidos como imágenes en el informe de manera tal que entren en el ancho de una hoja A4 
%sin tener que rotarla. No se aceptarán diagramas en archivos sueltos. 


% \footnote{Mensaje debajo}

\begin{figure}[H]
  \centering
  \includegraphics[width=0.98\textwidth]{diagrama_clase01.png}
  \caption{\label{fig:class01}Diagrama Asocioaciones entre Clases.}
  \end{figure}

  En el diagrama de la figura \ref{fig:class01} se observa que cada AlgoFix tendra por lo menos un pintor que siempre existirá 
  junto con su herramienta, la cual puede ser Pincel o Rodillo. 
  Esta tendrán el mismo comportamiento pero con sus propias propiedades. 
  
  En particular, la herramienta definirá las horas que tendra que trabajar, los Litros de pintura a gastar y 
  el descuento que hace (si es que hace) al pintar mas de 40 $M^2$ el Pintor. 

\begin{figure}[H]
\centering
\includegraphics[width=1\textwidth]{diagrama_clase02.png}
\caption{\label{fig:class02}Diagrama de dependencias debiles con la clase Pintura.}
\end{figure}

En la figura \ref{fig:class02} se representan las dependencias debiles que tiene las distintas clases con la clase Pintura. Se separo del diagrama \ref{fig:class01} de modo que
 de aporte a una mayor claridad. Adicionalmente, como ya se incluyeron las distintas clases en el diagrama de la figura \ref{fig:class01} 
entonces se abstraen unicamente los metodos que hacen referencia al objeto Pintura.\newline

Dado que distintas pinturas requieren distintos trabajos, del diagrama se observa que el Pincel y Rodillo le preguntan cuantas manos deben pasar y el Pintor le pregunta
cuanto es el costo de la pintura segun los litros que se usó. Esto es debido a que la Pintura tiene comportamiento propio definido el cual solamente ésta conoce.


% Explicaciones sobre la implementación interna de algunas clases que consideren que puedan llegar a resultar interesantes.
\section{Detalles de implementación}\label{sec:implementacion}
\begin{itemize}  
  \item Se tomaran los supuestos hechos en la sección \ref{sec:supuestos} como invariantes en todo momento del programa,
  por tanto las propiedades propias de cada objeto se asignan en su inicializacion en el momento de su creacion y no cambiarán.

  \item Al refactorizar el codigo se observó que el comportamiento de \textbf{Pincel} y \textbf{Rodillo} era idéntico en todo excepto en el numeroManos. 
  Por tanto se tomó la decisión de crear una clase abstracta \textbf{Herramienta} que contenga el comportamiento en común y luego cada uno define el numeroManos 
  necesario para pintar, según la pintura con la cual se trabaje.
\end{itemize}


\subsection{Nota de los tests}
De las pruebas integradoras, se extrajeron cada una de las pruebas unitarias y se pusieron en AlgoFixTestUnitarios 
de modo de simplificar el panorama y enfocar unicamente la atencion en resolver un problema a la vez. 
De este modo, su proposito es cumplir con las pruebas por separado para que luego pasen las pruebas integradoras. \newline

Adicionalmente se creo una clase de prueba por cada clase creada para resolver el trabajo practico
en las cuales se pone a prueba el comportamiento esperado del objeto en tiempo de ejecucion. Por tanto
las pruebas de AlgoFix estan orientadas a que el objeto AlgoFix se comporte como es de esperar en tiempo de ejecucion, 
y es distinto de las otras pruebas.


% Explicación de cada una de las excepciones creadas y con qué fin fueron creadas.
\section{Excepciones}\label{sec:excepciones}
\begin{description}
\item[Exception \underline{NumeroMetrosInvalido:}] Se lanza la excepción al pasarle un \textbf{numeroMetros} invalido al Pintor al calcular el presupuesto.
\item[Exception \underline{PrecioPorLitroInvalido:}] Se lanza la excepción al crear una Pintura nueva y pasarle un \textbf{precioPorLitro} invalido.
 Al crear la pintura se asume que se cumplen los supuestos \ref{sec:supuestos} y se respeta el numero de manos correspondiente para cada herramienta.
\item[Exception \underline{ValorPorHoraInvalido:}] Se lanza la excepción al crear un Pintor nuevo y asignarle un \textbf{valorPorHora} de trabajo invalido.
\end{description}

% Mostrar las secuencias interesantes que hayan implementado. Pueden agregar texto para explicar si algo no queda claro.
\section{Diagramas de secuencia}\label{sec:diagramasdesecuencia}

\subsection{test01ObtenerPresupuestoPintor}
En la figura \ref{fig:seq01} se muestra el diagrama de secuencia de los pasos para obtener el pintor de menor presupuesto.
El siguiente diagrama esta basado en el test01, sin embargo se devuelve el propio pintor ya que el actor no es SUnit y se 
trata de modelar un caso real donde el interés es calcular el pintor de menor presupuesto.

\begin{figure}[H]
\centering
\includegraphics[width= 1.1\textwidth]{diagrama_secuencia01.png} 
\caption{\label{fig:seq01}Comportamiento esperado para obtener pintor de menor presupuesto.}
\end{figure}

\underline{Observaciones}
\begin{itemize}
  \item Se recortó el nombre del metodo \textbf{presupuestoMasBaratoParaPintarM2ConPintura}  a  \textbf{presp...M2Pintura} para darle mayor claridad al diagrama.
  \item Cada pintor es responsable de conocer el precio que cobrará por el trabajo; por tanto, al buscar el menor presupuesto se busca el pintor que cobre menos preguntandole a cada uno.
  \item Los pasos que hace el pintor para calcular su presupuesto se abstrayeron en la figura \ref{fig:seq02} para darle claridad al diagrama.
\end{itemize}

\begin{figure}[H]
\centering
\includegraphics[width=\textwidth]{diagrama_secuencia02.png}
\caption{\label{fig:seq02} Diagrama de Secuencia de calcularCostoM2ConPintura(metros: int, pintura: Pintura).}
\end{figure}

\subsection{test04ObtenerDescuentoMas40M2}

El siguientes diagramas modela el test04 de las pruebas unitarias del Pintor. Al pintar mas de
40 $M^2$ usando el Pincel y la pintura Alba entonces aplica un descuento del 50\%. Al igual que la figura \ref{fig:seq01} se 
va a partir en 2 partes la interacción para destacar los aspectos más importantes.
\begin{figure}[H]
  \centering
  \includegraphics[width=1.0\textwidth]{diagrama_secuencia03.png}
  \caption{\label{fig:seq03}Diagrama de Secuencia del test04.}
  \end{figure}

  \begin{figure}[H]
    \centering
    \includegraphics[width=1.0\textwidth]{diagrama_secuencia04.png}
    \caption{\label{fig:seq04}Diagrama de Secuencia aplicando descuento.}
    \end{figure}

  Se puede observar que el comportamiento es idéntico al de la figura \ref{fig:seq02} 
  con la excepción que ahora aplica el descuento. El objeto descuento conoce cuánto descuento debe de aplicar según la herramienta 
  tambien decide si se puede aplicar o no.


  \section{Conclusión}
  
  En el presente trabajo practico se modeló un sistema que permite encontrar el pintor de menor presupuesto utilizando el paradigma de la programacion orientada a objetos. 
  Se observó que cada objeto creado tiene un proposito y es encargado de responder a su comportamiento esperado. Es importante destacar la 
  importancia del cumplimiento del contrato por parte del objeto cliente ya que en caso de no cumplirse se lanzara alguna excepción 
  debido a que el objeto receptor no sabra cómo responder al mensaje. Los diagramas en la figura \ref{fig:class01} y \ref{fig:class02} modelan las dependencias 
  fuertes y debiles entre objetos. \newline

  La resolución del trabajo practico usando este paradigma deja las puertas abiertas para una posible reimplementación o mejora de los métodos sin 
  necesidad de volver a diseñarlo de nuevo. Junto con la ayuda de las distintas pruebas hechas, permite comprobar que se mantenga el mismo 
  comportamiento implementado en este trabajo practico.

\newpage
\section{Referencias}

\begin{itemize}
  \item UML gota a gota. Martin Fowler con Kendall Scott
  \item Texto de apoyo conceptual de Algoritmos y Programación III Facultad de Ingeniería de la Universidad de Buenos Aires. Carlos Fontela 2018
  \item Diagramas de Secuencia: \url{https://plantuml.com/sequence-diagram}
  \item Diagramas de Clase: \url{https://plantuml.com/class-diagram}
  \item Diagrama de Clase: \url{http://ogom.github.io/draw_uml/plantuml/}
  \item UML si o si: \url{https://campus.fi.uba.ar/mod/lesson/view.php?id=98901&pageid=2105&startlastseen=no}
  \item Documentacion para la herramienta LaTex: \url{https://www.overleaf.com/learn/latex/Main_Page}
\end{itemize}

\end{document}